\documentclass[12pt]{article}
\usepackage{sbc-template}
\usepackage{graphicx}
\usepackage{amsmath}
\usepackage{subfigure}
\usepackage{times,amsmath,epsfig}
\usepackage{graphicx,url}
  \makeatletter
  \newif\if@restonecol
  \makeatother
  \let\algorithm\relax
  \let\endalgorithm\relax
\usepackage[lined,algonl,ruled]{algorithm2e}
\usepackage{multirow}
\usepackage[brazil]{babel}
\usepackage[utf8]{inputenc}
\usepackage[pdftex]{hyperref}

\sloppy

\title{Algoritmos e Estruturas de Dados 3 \\ Trabalho Prático 4 \\
\huge{CDFs! \\ Ordenação em memória externa}}
\date{November 22, 1963}


\author{André Taiar Marinho Oliveira}


\address{Departamento de Ciência da Computação -- Universidade Federal de Minas Gerais (UFMG)
\email{taiar@dcc.ufmg.br}
\\
\\ Novembro de 2010
}

\begin{document}

\maketitle

\begin{resumo}
Muitas aplicações em computação necessitam trabalhar com volumes de dados maiores do que
o que a memória principal (ou memória interna) das máquinas suporta. Tais aplicações precisam
fazer uso dos discos dos computadores que fornecem uma quantidade de espaço ordens de grandeza 
maior e suportam o volume de dados a ser computado, pagando para isso o preço da eficiência (dados 
em disco demoram ordens de grandeza de tempo a mais para serem acessados do que em memória interna).
Neste trabalho aplicaremos um método de ordenação em memória secundária para resolver o problema
das CDFs (\textit{Cumulative Distribution Function}) com um volume extenso de dados.
\end{resumo}

\section{Introdução}

A função de probabilidade acumulada (ou CDF) descreve completamente a
distribuição da probabilidade de uma variável aleatória de valor real $X$. 
Para cada número real $X$, a CDF é:

$F(x) = P(X =< x)$

Isso significa que, dado um número qualquer $X$ a CDF indica a probabilidade dos
outros valores $X$ serem menores ou iguais a $x$.

O algoritmo utilizado nesse trabalho para calcular a CDF de um conjunto de números em ponto flutuante é:

\begin{algorithm}[h!]
\begin{footnotesize}

$D \longleftarrow $ conjunto muito grande de números (floats)\;

\tcp{Esta é a função mais importante deste trabalho.}
\textbf{ORDENA}$(D)$\; 

$count \longleftarrow$ 0\;
$current \longleftarrow$ $D[0]$\;
$len \longleftarrow$ TAMANHO$(D)$\;

\ForEach{$n \in D$}{
  \If{$n \neq current$}{
    $current \longleftarrow n$\;
    $prob \longleftarrow count/len$\;
    ESCREVE$(current, prob)$\;
  }
  $count \longleftarrow count + 1$\;
}
$n \longleftarrow D[len - 1]$\;
ESCREVE$(n, 1)$\;

\caption{Cálculo da CDF}
\end{footnotesize}
\end{algorithm}


\section{Solução Proposta}
\label{solucao_proposta}

falar sobre a representacao por grafos.

\subsection{Solução Ótima}


\subsection{Heurística}
A heurística utilizada neste trabalho é baseada em uma heurística clássica da literatura
conhecida como \textbf{Heurística do Vizinho Mais Próximo}. Esta, funciona assim:

\begin{algorithm}[h!]
\begin{footnotesize}
  $G \longleftarrow $ grafo que contém as distâncias rodoviárias e aéreas entre todas as 
  capitais do Brasil\;
  $c \longleftarrow $ lista de cidades não visitadas\;
  $p \longleftarrow $ percurso gerado pela heurística\;
  $CidadeAtual \longleftarrow $ cidade em que o percurso vai começar\;
  $p_{1} \longleftarrow CidadeAtual$\;
  \ForEach{$c$ não visitada em $G$}{
    \If{Distância de $c_{i}$ para $CidadeAtual$ é a menor de todas as não visitadas}{
      $CidadeAtual \longleftarrow c_{i}$\;
      $p \longleftarrow CidadeAtual$\;
    }
  }
  \Return{$p$}
\caption{Heurística do Vizinho Mais Próximo}
\end{footnotesize}
\end{algorithm}

É um algoritmo guloso que itera fazendo as melhores escolhas locais, ou seja:
pega sempre a menor distância entre a cidade atual e uma cidade do conjunto de cidades ainda
não visitadas disponíveis, até completar o percurso. Ela serve para resolver o PCV simples. 
Utilizamos uma adaptação da heurística do vizinho mais próximo para resolver o nosso problema.
Os dados fornecidos são os mesmo do problema anterior acrescidos das restrições já discutidas na 
Introdução. Os passos são os seguintes:

\begin{itemize}
  \item À partir da cidade inicial/atual, verificar se existe alguma região que tem necessidade
  de ser visitada
  
  %acabar de descrever e fazer este algoritmo
\end{itemize}

% algoritmo da heuristica:
% modificacao na heuristica do vizinho mais proximo
% 1 - esta na cidade inicial e escolhe qual a proxima baseada na distancia de carro
% 2 - procede da forma 1 ate o tempo limite para ir para alguma regiao ser alcançado
% 3 - com o limite alcancado, escolhe qual a menor distancia da regiao deve ir de carro
% 4 - procede dessa forma ate que o caminho inteiro seja percorrido de carro
% 5 - se os dias foram suficientes para que o caminho fosse feito de carro entao esse deve ser escolhido
% 6 - se nao for, substituir as menores distancias de carro por distancias de aviao
% 7 - seguir dessa forma ate que o numero de dias seja alcancado
% 8 - e se o dinheiro nao for o suficiente?
% --------------> isso significa que a heuristica nao funcionou para esta instancia por que 
% --------------> o caminho escolhido nao foi pequeno o suficiente. ou ainda, que o caminho nunca poderia
% --------------> ser completado com o dinheiro disponivel nos dias disponiveis 

\begin{algorithm}[h!]
\begin{footnotesize}
  \ForEach{Documento}{
    \ForEach{termo válido}{
      \If{o termo já apareceu em algum texto}{
        \If{o termo já apareceu neste texto}{
          incrementa o contador de ocorrências correspondente ao termo no texto;\
        }
        \Else{
          insere o texto na lista e contabiliza o contador de ocorrências para
          este termo;\
        }
      }
      \Else{
        insere o nó referente ao termo, insere o texto na lista e contabiliza o
        contador de ocorrências para este termo;\
      }0.000012 0.000000

    }
  }
\caption{Leitura do índice}
\end{footnotesize}
\end{algorithm}

Em uma aplicação real, muitos passos podem ser feitos para determinar
efetivamente o que é e o que não é um termo relevante e, dessa forma, poupar
processamento e espaço em processos de indexação. Alguns desses passos
correspondem à análise de Stop Words \footnote{Stop Words são palavras consideradas sem valor
semântico para a análise de tópico de um texto (como artigos e preposições, por
xemplo). \href{http://searchenginewatch.com/2156061}{Referência}.}, remoção de
prefixos e sufixos das palavras \footnote{É um processo que simplifica as
palavras removendo seu prefixo e sufixo (caso tenha) e dessa forma reconhece o
padrão de formação daquela palavra.}, reconhecimento sintático, entre outras. 
Em nosso sistema, a única análise feita é quanto ao número de caracteres que a 
palavra contém. Para uma aplicação padrão da indexação, utilizei o mínimo de 3 
caracteres por palavra (parâmetro que será variado nas avaliações experimenais).

Após indexar todos os textos, conseguimos obter vários parâmetros úteis para
expressar a relevância das palavras-chave neste contexto. O primeiro aspecto a ser
levado em consideração é o número de ocorrências de uma palavra. Como sugerido
na especificação, assumi que, para um termo ser considerado relevante ele
deveria aparecer na minoria dos textos apresentados (aparecer em menos de 50\%
dos textos) aumentando o índice de discriminação que ele potencialmente tem.
Esse parâmetro foi colocado (assim como o tamanho mínimo dos termos) de forma a
ser facilmente modificado para obtermos melhores análises experimentais.

Após termos uma referência sobre quais palavras serão potencialmente relevantes
para analisarmos as semelhanças entre os textos, podemos reconstruir novamente
um índice. Dessa vez, eu analiso cada texto e incorporo ao seu índice apenas as
palavras relevantes, colocando cada índice em um vetor que terá o índice de um texto 
em cada posição.

Com estes elementos calculados podemos partir para a finalização do trabalho.
Para retornar as palavras chaves de cada texto, basta percorrer o índice
individual de cada texto. Porém, como as palavras chaves precisam ser retornadas
devemos primeiramente ordená-las de acordo com as suas ocorrências naquele
texto. Para isso, lemos o índice e armazenamos os termos e suas ocorrências um
um vetor que será posteriormente ordenado decrescentemente, obtendo dessa forma
as palavras-chave de um texto ordenadas por sua relevância.

O segundo item a ser retornado pelo programa é um arquivo contendo a lista dos
textos analisados se os texos mais semelhantes à ele. O número de textos a ser
retornado é arbitrário. Nas execuções do trabalho, retornamos quantidades de 5 a
8 indicações de textos.

\begin{algorithm}[h!]
\begin{footnotesize}
  \ForEach{Documento}{ 
    $PalavrasChave \longleftarrow$ recebe as palavras-chave do Documento e o seu
    número de ocorrências naquele texto\;
    $OrdenaOcorrenciaDecrescente(PalavrasChave)$\;
    \textbf{Imprime} lista de palavras-chave\;
    \ForEach{palavras-chave do Documento}{
      $Ocorrencias \longleftarrow$ vetor de textos e numero de ocorrencias 
      da palavra-chave por texto\;
      \ForEach{Documento em que a palavra-chave ocorreu}{
        $PotencialDeSemelhanca[Documento de Ocorrencia] + \longleftarrow$
        ocorrencias da palavra-chave no Documento de Ocorrencia\;
      }
    }
    $OrdenaPotenciaDecrescente(PotencialDeSemelhanca)$\;
    \textbf{Imprime} lista de $\textbf{N}$ Documentos mais similares\;
  }
\caption{Identificação de textos semelhantes}
\end{footnotesize}
\end{algorithm}

Encerrada a explicação sobre os principais algoritmos e as estruturas de dados
utilizados, e como eles se combinaram na solução do problema, segue abaixo uma
breve análise sobre a análise de complexidade dos principais algoritmos e mais
alguma explicação sobre as estruturas que eventualmente tenha faltado acima.

\subsection{Análise da solução}

\subsubsection{Ordenação}
O algoritmo de ordenação utilizado foi o Quicksort que tem ordem de complexidade
$n(\log(n))$ para casos médios.

\subsubsection{Dicionário}
Armazena cada termo, o documento em que ocorreu e quantas
vezes ocorreu naquele documento. O Dicionário foi organizado como uma árvore 
binária de pesquisa em que cada termos e suas ocorrências estão em um nó da 
árvore. Dessa forma, cada nó ainda conta com 2 ponteiros (um que aponta para 
a sub-árvore da direita e um que aponta para a sub-árvore da direita).

\begin{algorithm}[h!]
\begin{footnotesize}
	termo\;
	lista de ocorrencias\;
\caption{Nó da Árvore}
\end{footnotesize}
\end{algorithm}

\begin{table}[ht]
  \caption{Análise de Complexidade do Dicionário}
  \centering
  \begin{footnotesize}
  \begin{tabular}{c c}
  \hline\hline
  Operação & Custo (casos médios) \\
  \hline
  Inserção na Árvore & $O(\log(n))$ \\
  Pesquisa na Árvore & $O(\log(n))$ \\
  Caminhamento na Árvore & $O(n)$ \\
  \hline
  \end{tabular}
  \end{footnotesize}
\end{table}

\subsubsection{Lista de Ocorrências}

Precisamos de uma lista de ocorrências para guardar todas as ocorrências que
variavam por termo e documentos. Cada nó do dicionário contém uma lista de
ocorrências.

\begin{algorithm}[h!]
\begin{footnotesize}
	identificador do documento\;
	ocorrências no documento\;
\caption{Item da lista}
\end{footnotesize}
\end{algorithm}

As operações em lista são sempre $O(n)$ para o pior caso.

\section{Código}

\subsection{Módulos}
O programa foi separado em quatro módulos:
\begin{itemize}
\item \textbf{Grafo:} contém todas as operações executadas sobre a 
estrutura de dados Grafo, que foi utilizada para armazenamento e cálculo
das distâncias entre as cidades.
\item \textbf{Entrada e Saída:} controla a leitura do arquivo de distâncias
e a escrita do resultado do programa na saída padrão. Também faz a validação dos 
argumentos de entrada pela linha de comando.
\item \textbf{Geografia:} contém dados como nomes das cidades (para sua identificação
e armazenamento de seu identificador único), descrição das regiões do brasil
segundo as capitais de seus estados e algumas operações sobre esses dados geográficos.
\item \textbf{Principal:} interagiu com todos os outros módulos retornando erro
quando alguma irregularidade era colocada, controlou o fluxo de execução do
programa.
\end{itemize}

\subsection{Entrada e saída}
Para não haver problemas com a leitura da linha de comando, foi utilizado um utilitário do
sistema Linux chamado \textbf{getopt} que permite uma leitura mais robusta dos argumentos
passados ao programa. O comando para execução do programa definido para a linha de entrada 
foi:
\begin{verbatim}
#: ./tp3 -a <A> -b <B> -c <C> -d <D> -e <E> -f <F> -g <G> 
\end{verbatim}
Sendo: 
\begin{itemize}
  \item \textbf{A} - Quantidade total de dias de viagem;
  \item \textbf{B} - Cidade em que a viagem começa (e consequentemente aonde deve terminar);
  \item \textbf{C} - Dinheiro disponível para o gasto com esta viagem;
  \item \textbf{D} - Quantidade máxima de dias que um candidato pode ficar sem visitar alguma
  região geográfica do Brasil.
  \item \textbf{E} - Preço por Km para viajar de Carro;
  \item \textbf{F} - Preço por Km para viajar de Avião; 
  \item \textbf{G} - Algoritmo a ser exevutado: solução ótima (\textit{otima}) ou a 
  heurística (\textit{heuristica}).
\end{itemize}

% falar sobre o arquivo de distancias
O arquivo de distâncias especifica a distância em Km entre cada uma das capitais brasileiras
no seguinte formato:
\begin{verbatim}
...
<A> <B> <C> <D>
...
\end{verbatim}
Sendo:
\begin{itemize}
  \item \textbf{A} - nome da cidade A;
  \item \textbf{B} - nome da cidade B; 
  \item \textbf{C} - distância rodoviária entre a cidade A a cidade B;
  \item \textbf{D} - distância aérea entre a cidade A e a cidade B. 
\end{itemize}

% falar sobre o formato da saida
A saída do programa é retornada na saída padrão do sistema no seguinte formato:
\begin{verbatim}
...
<A> <B> <C>
...
\end{verbatim}
Sendo:
\begin{itemize}
  \item \textbf{A} - número do dia em que o candidato chegou à cidade;
  \item \textbf{B} - nome da cidade em que o candidato chegou; 
  \item \textbf{C} - meio de transporte utilizado pelo candidato para ir até a próxima cidade.
\end{itemize}

\subsection{Compilação}
O compilador utilizado neste trabalho foi o GCC (adotado como padrão para a disciplina) e 
o comando para compilar o programa através do GCC é:
\begin{verbatim}
  #: gcc -o tp3 tp3.c io.c io.h geografia.c geografia.h grafo.c grafo.h
\end{verbatim}
caso estejam todos os arquivos dentro do mesmo diretório.

Como pedido na especificação, foi feito um arquivo Makefile com o qual é possível compilar
o programa com o comando: 
\begin{verbatim}
  #: make all
\end{verbatim}

E podemos compilar e executar o programa com o comando:
\begin{verbatim}
  #: make run
\end{verbatim}
que fará, além da compilação, com que o programa execute com uma entrada arbitrária.

\section{Avaliação Experimental}
\label{avaliacao_experimental}

O que pudemos avaliar experimentalmente com a execução do programa é que apesar de ter 
encontrado bem as palavras-chave de cada texto (os termos considerados foram realmente 
relevantes), o algoritmo para clusterizar os textos não foi tão eficiente assim.

Aparentemente, textos com muitas palavras foram considerados muito mais que outros textos 
menores. Esta falha provavelmente se encontra no algoritmo do cálculo do potencial de 
semelhança entre os textos, descrito em alto-nível no algoritmo 2. Em uma próxima versão 
será interessante normalizar matemáticamente essa contagem para evitar esse tipo de 
discrepância de resultados.

Como não fiz medições do tempo de execução do programa para diversas entradas, a avaliação
da velocidade do programa para diferentes entradas será omitida.

\section{Conclusão}
\label{conclusao}

Neste trabalho construímos um sistema que permite separar por critérios de similaridade
(clusterizar) um conjunto de textos arbitrário através da identificação de 
tópicos/palavras-chave desse texto.

O trabalho conseguiu cumprir o propósito de praticar a linguagem C e o estudo de um 
problema mais complexo do que o de costume. A utilização massiva de estruturas de pesquisa
foi uma novidade. Uma decisão extremamente importante para o trabalho foi a de utilizarmos 
uma estrutura em forma de árvore binária ao invés de outras que poderíamos utilizar como,
por exemplo, tabelas hash o que provavelmente fez o programa executar com uma velocidade 
maior e gastando uma quantidade menor de espaço em memória.

A análise da variação da qualidade da similaridade avaliada pelo programa é uma avaliação 
subjetiva pois depende da leitura e interpretação dos textos retornados por parte de uma
pessoa (melhor avaliadora de similaridade de conteúdo possível). Porém, ao analisarmos a 
ocorrência de palavras chaves vemos que o sistema encontrou palavras que tinham realmente 
a ver com o conteúdo daquele texto e que não eram tão decorrentes assim nos outros (tinham 
certo caráter discriminativo entre os textos analisados).

Algumas melhorias que poderiam ser consideradas neste trabalho são:
\begin{itemize}
\item Um melhor tratamento dos textos de entrada (eliminando acentuação e outros 
caracteres inválidos dentro do programa);
\item Utilização de Stop Words, retirada de prefixo e sufixo e outros tipos de técnicas
aplicadas em recuperação de informação com o objetivo de maximizarmos o aproveitamento dos
termos contidos no índice;
\item Uma melhor análise de complexidade de espaço e tempo de execução do programa, talvez
a utilização de uma outra estrutura de dados para o índice seja mais eficiente em termos 
de execução;
\item Uma modelagem mais complexa da análise de frequência das ocorrências de termos 
relevantes nos documentos analisados;
\item Uma modelagem mais complexa do cálculo de similaridade entre os textos a partir de 
suas/seus palavras-chave/tópicos.
\end{itemize}

\end{document}
